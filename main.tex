\documentclass[a4paper,12pt]{report}
\usepackage[top=2cm, bottom=2cm, outer=2cm, inner=6.2cm, heightrounded, marginparwidth=5cm, marginparsep=1cm]{geometry}
\usepackage{hyperref}
\usepackage{marginnote}
\usepackage{xepersian}

\settextfont{XB Zar}

% \reversemarginpar

\begin{document}
\TeX
(بخوانید تِک) یک برنامه کاربردی است که توسط 
 \href{http://en.wikipedia.org/wiki/Donald_Knuth}{دونالد کنوث}
\marginpar{
کنوث 
\includegraphics[width=1cm]{../Pic/don.png}
 اولین بار بخاطر کیفیت نامناسب سری کتاب‌های 
\lr{Art of Programming}
شروع به طراحی تک کرد.
 }
با هدف حروف‌چینی متن عادی و ریاضی، طراحی شده است. 
وی در سال ۱۹۷۷ شروع به نوشتن 
\TeX
و در سال ۱۹۸۲ آن را منتشر کرد. سال ۱۹۸۹ با اضافه کردن پشتیبانی از حروف ۸ بیتی تک برای دیگر زبان‌ها عرضه شد. نسخه ۳.۱۴۱۵۹۲ (که به عدد $\pi$ میل می‌کند) آخرین نسخه‌ای است که به عنوان نسخه پایدار تک انتخاب شده است.

\TeX
در زبان یونانی به معنی هنر و همچنین به معنی تکنولوژی است و به دلیل ظرافت مفهوم‌های این واژه، کنوث این نام را برگزید.

\section*{چرا \TeX؟}
\begin{itemize}
\item
تک یک نرم‌افزار آزاد است. 
\marginpar{\lr{\TeX $\:$ is free (as in beer), and free (as in speech)}}
\item
تک روی زیبایی و ساخت‌یافته بودن خروجی تاکید دارد. به همین دلیل متن و قالب نوشتن را از هم جدا کرده است.
\marginpar{
این کار تقریبا شبیه طراحی دو فایل  \lr{.css} و \lr{.html} برای یک صفحه وب است.
}
 بنابراین، شما برای تغییر قالب نوشته‌تان کافی است که فایل قالب را تغییر دهید و به متن‌تان کاری ندارید. بدین ترتیب از ابتدای نوشتن نیازی به طراحی سبک نوشتن ندارید و حتی طراحی آن را می‌تواند شخص دیگری به صورت موازی با شما انجام دهد.
\item
کیفیت خروجی تک بسیار بالاست. 
\marginpar{
\includegraphics[width=3.5cm]{../Pic/boxes-with-text-and-math.png}
}
هدف از طراحی تک، حروف‌چینی اتوماتیک بوده است و می‌تواند این کار را بسیار زیبا، حتی در فایل‌هایی که فرمول‌ها و جداول پیچیده‌ای دارند انجام دهد و نیازی نیست که شما حتما کاربر حرفه‌ای باشید تا با چنین کیفیتی حروف‌چینی انجام دهید.

\item
تک در انجام عملیات و تولید فایل مورد نظر  سریع است.
\marginpar{
\includegraphics[width=4.2cm]{../Pic/parabola-plot.png}}
 معمولا نرم‌افزارهایی که کند هستند کاربر را به سرعت خسته می‌کنند و استفاده کردن از آنها کسل کننده است. خوش‌بختانه، تک سرعت اجرای زیادی، حتی در کتاب‌های با تعداد صفحات بالا و با تعداد زیادی عکس، دارد (تجریه شخصی: سرعت اجرای آن در توزیع‌های لینوکسی به مراتب از ویندوزی بیشتر است).

\item
تک پایدار است. کنوث موتور اصلی حروف‌چینی تک را دیگر تغییر نمی‌دهد و آن را به تثبیت رسانده است. بنابراین، فایل‌های شما برای همیشه با تمامی‌ نسخه‌های بعدی تک سازگارند.

\item
تک قابل توسعه است. پایدار بودن موتور اصلی به این معنی توسعه ناپذیری آن نیست.  
هر شخصی ‌می‌تواند بنا به نیاز‌های خود ماکروهای مورد نظر خود را  نوشته و به تک بیافزاید و از آن‌ها استفاده کند.
\marginpar{
در نوشتن متن‌هایتان می‌توانید قسمت‌های مختلف‌ آن را در فایل‌های مختلف قرار داده و در نهایت در یک فایل همه آنها را با دستورات
 \lr{include}
 و
 \lr{input}
 وارد کرده و متن کلی را بسازید.
}
 به عنوان مثال \LaTeX شناخته‌شده‌ترین روش استفاده از تک است که با اضافه کردن ماکروهای زیادی استفاده از آن را بسیار آسان‌تر کرده است. به عنوان مثال شماره‌گذاری عکس‌ها، مثال‌ها و قضیه‌ها و مراجع  و بخش‌ها و ... به صورت خودکار انجام می‌گیرد و شما نگرانی از جابجایی آنها و یا اشتباه شدن ارجاع به آنها را ندارید. امروزه کمتر کسی از خود تک مستقیما استفاده می‌کند و اکثرا با لاتک (که آخرین نسخه آن \LaTeXe است)‌ کار می‌کنند. 

\item
تک قابل اعتماد است.
\marginpar{
ماکرو‌های تک نمی‌توانند ویروس باشند، و فایل‌های تک را از طریق ایمیل‌تان  بدون ترس از ارسال اطلاعاتتان می‌توانید دریافت کنید.
}
 سال‌های زیادی از پایدار شدن موتور اصلی تک می‌گذرد و تقریبا می‌توان مطمئن بود که مشکل نرم‌افزاری در آن وجود ندارد و بدون نگرانی از دست رفتن اطلاعاتتان می‌توانید با آن کار کنید. کنوث طبق روال معروف خودش که برای یابندگان اشکالات کتاب‌ها و نرم‌افزارهایش جایزه تعیین می‌کند، برای تضمین صحت عملکرد تک، جایزه ۳۲۷.۶۸دلاری (معادل با $2^{15}$ سنت) تعیین کرده است.

\item
تک مستقل از سیستم عامل است 
\marginpar{
میک‌تک توسط یک نفر توسعه می‌باید و فقط تحت ویندوز است، ولی تک‌لایو توسط یک گروه توسعه می‌یابد و برای هر سه سیستم‌عامل لینوکس، ویندوز و مک ارائه می‌شود.
}
و در سیستم‌عامل‌های ویندوز، لینوکس، و مک اجرا می‌شود. نسخه miktex فقط مخصوص ویندوز است اما نسخه texlive که برتری‌های بیشتری نسبت به miktex دارد
روی هر سه سیستم‌عامل نصب می‌شود، در نتیجه استاندارد یکسانی برای متن شما در هر سه سیستم‌عامل رعایت می‌شود. بنابراین، چند نفر بدون نگرانی از نوع سیستم‌عامل خود می‌توانند با هم روی یک متن کار کنند.

\item
ورودی تک یک فایل ساده با پسوند \lr{.tex} است.
\marginpar{
بدلیل ساده بودن فایلهای تک، می‌توانید در ویرایشگر دلخواه خودتان متن مورد نظر را بنویسید و در ترمینال آن را کامپایل کرده و خروجی دلخواهتان را دریافت کنید.
}
 ساده بودن فایل، مزیت‌های زیادی را برای تک به ارمغان می‌آورد، از جمله حجم بسیار کم فایل‌های آن،  که این حجم کم موجب راحت و سبک بودن نرم‌افزار هنگام کار با آن می‌شود و همچنین کتاب‌های با تعداد صفحات بالا حجم اندکی را اشغال می‌کنند و به راحتی می‌توانید آن‌ها را جابجا کنید. همچنین کار گروهی روی یک متن مشکلات خاصی ندارد،‌به علاوه با استفاده از نرم‌افزارهای کنترل نسخه مانند git به راحتی می‌توانید روند نوشتن متن را مدیریت کنید.

\item
\marginpar{
اهمیت توضیحات شاید برای فایل‌های کوچک و نوشتن فایل‌های انفرادی زیاد به چشم نیاید، اما برنامه‌نویسان حرفه‌ای که کارهای گروهی بزرگ انجام داده‌اند، مزیت نوشتن توضیحات در کار گروهی را به خوبی درک می‌کنند.
}
در فایل ورودی تک می‌توانید توضیحات اضافه کنید. 
\item
استاندارد بیشتر مجلات، کتاب‌های علمی، دانشگاه‌ها، همچنین وب‌سایت‌هایی مانند ویکی‌پدیا تک است و دانستن آن تقریبا جزء ملزومات برای یک دانشجو است.

\item
از فایل ورودی به تک می‌توان خروجی‌های مختلفی مانند \lr{pdf}،
\lr{html}،
 و PostScript دریافت کرد.

\item
GUID
ندارد. 
\marginpar{
به طور مثال نویسنده ویروس Melisa به همین روش شناسایی شد.
}
\lr{Global Unique Identification}
روشی برای ردگیری نویسنده یک فایل است که در بیشتر فایل‌ها بدون اطلاع‌رسانی خوب به کاربران این شناسه ذخیره می‌شود، اما تک این کار را انجام نمی‌دهد.

\end{itemize}

\section*{\TeX برای چه کسانی مناسب نیست؟}

تک برای همه نیست ولی می‌تواند برای شما باشد و از قدرت آن بهره ببرید. در ادامه مشکلات کار با تک نیز بررسی شده و اگر فکر می‌کنید که این مشکلات تک را نمی‌توانید تحمل کنید و مزیت‌های آن را به خاطر این مشکلات نادیده می‌گیرید، تک برای شما مناسب نیست.

\begin{itemize}
\item

یادگیری تک زمان‌بر است. 
\marginpar{
به لطف اینترنت و عمر به نسبت زیاد تک، مشکلات زیادی که کاربران با آن مواجه می‌شوند معمولا در اینترنت درباره آن بحث شده و راه‌کار مناسبی برای آن ارائه شده است.
}
یکی از بزرگ‌ترین ایراداتی که به تک می‌گیرند، زمان‌بر بودن دوره یادگیری آن است و موارد زیادی دیده شده کسانی که اولین بار از آن استفاده کرده‌اند در میانه راه خسته شده و می‌گویند که دیگر از این نرم‌افزار استفاده نمی‌کنند، اما بعد اتمام کار و مشاهده خروجی، نظر بیشتر آنها برمی‌گردد.

\item
جدا کردن فایل تنظیمات قالب از متن، کار با جدول‌ها و تصاویر را به نسبت نرم‌افزار‌هایی که این کار را انجام نمی‌دهند مانند
\lr{libreOffice Writer} 
یا
 \lr{Microsoft Office Word}
، که به نرم‌افزارهای WYSIWYG 
\marginpar{
\begin{center}
\lr{What You See Is What You Get}
\end{center}
}
معروفند، سخت‌تر می‌کند و گاهی فهمیدن نحوه انجام کاری که می‌خواهید انجام دهید دشوار است.

\item
اگر در موقع نوشتن اشتباه کنید متن خروجی تولید نمی‌شود و حتما باید آن اشتباهات را اصلاح کنید. البته شماره خط و توضیحی در مورد آن به شما نشان داده می‌شود و در بیشتر موارد توضیحات ارائه شده مفید و کارا هستند.
\end{itemize}

\section*{فارسی}
شاید یکی از مهمترین سوالاتی که برای ما ایرانی‌ها مطرح می‌شود این است آیا تک از زبان فارسی پشتیانی می‌کند و می‌توان فایل‌های فارسی با آن تولید کرد؟ متاسفانه هیچ حمایت مالی برای این کار در ایران صورت نگرفته است، ولی خوش‌بختانه، افراد زیادی برای پاسخ آری به این سوال زحمات فراوانی بدون هیچ دست‌مزدی کشیده‌اند. از جمله این کارها می‌توان به فارسی تک اشاره کرد که در دانشگاه شریف سال  ۱۹۹۲ به مدیریت دکتر محمد قدسی و مهندس اسفهبد 
\marginpar{
لیست کامل اسامی که در این پروژه حضور داشتند را در وب‌سایت این نرم‌افزار می‌توانید مشاهده نمایید.
}
نوشتن آن آغاز شد و تا سال ۲۰۰۷ نسخه‌های مختلفی از آن ارائه شد. و همچنین 
\XePersian
(بخوانید زی‌پرشین) بسته آماده‌ای برای فارسی نویسی است که امروزه در کامپایلر‌های تک موجود است. دکتر وفا خلیقی زحمت نوشتن و پشتیبانی این پروژه را به عهده گرفته‌اند. این بسته آماده، بدلیل توانایی استفاده از فونت‌های مختلف و به روزرسانی‌هایی که برای آن ارائه می‌شود امروزه از محبوبیت بیشتری برخوردار است. در این زمینه آموزش‌های زیادی از نحوه تنظیمات تا شروع کار با آن را می‌توانید در اینترنت بیابید، 
\marginpar{
برای تست و کار با تک و حتی استفاده از بسته زی‌پرشین می‌توانید به وب‌سایت writelatex.com مراجعه نمایید.
}
که یکی از بهترین وب‌سایت‌ها \lr{www.parsilatex.com} است، که علاوه بر فروم تعبیه شده جهت بحث گفت و گو، آموزش‌ها و نرم‌افزارهای مورد نیاز جهت نصب و راه‌اندازی تک، فونت‌های زیبا و نمونه فایل‌های مختلف را نیز ارائه می‌دهد.
 برای دریافت تک‌لایو می‌توانید به مخزن http://ctan.yazd.ac.ir/ که توسط استادهای دانشگاه یزد حمایت می‌شود، مراجعه نمایید.


\end{document}